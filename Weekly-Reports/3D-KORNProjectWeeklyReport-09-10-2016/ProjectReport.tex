% !TEX TS-program = pdflatex
% !TEX encoding = UTF-8 Unicode

% This is a simple template for a LaTeX document using the "article" class.
% See "book", "report", "letter" for other types of document.

\documentclass[11pt]{article} % use larger type; default would be 10pt

\usepackage[utf8]{inputenc} % set input encoding (not needed with XeLaTeX)

%%% Examples of Article customizations
% These packages are optional, depending whether you want the features they provide.
% See the LaTeX Companion or other references for full information.
\usepackage{float}
%%% PAGE DIMENSIONS
\usepackage[margin=1.0in]{geometry} % to change the page dimensions
\geometry{a4paper} % or letterpaper (US) or a5paper or....
% \geometry{margin=2in} % for example, change the margins to 2 inches all round
% \geometry{landscape} % set up the page for landscape
%   read geometry.pdf for detailed page layout information

\usepackage{graphicx} % support the \includegraphics command and options

% \usepackage[parfill]{parskip} % Activate to begin paragraphs with an empty line rather than an indent

%%% PACKAGES
\usepackage{booktabs} % for much better looking tables
\usepackage{array} % for better arrays (eg matrices) in maths
\usepackage{paralist} % very flexible & customisable lists (eg. enumerate/itemize, etc.)
\usepackage{verbatim} % adds environment for commenting out blocks of text & for better verbatim
\usepackage{subfig} % make it possible to include more than one captioned figure/table in a single float
% These packages are all incorporated in the memoir class to one degree or another...
\setlength{\parskip}{0.5em}

%%% HEADERS & FOOTERS
\usepackage{fancyhdr} % This should be set AFTER setting up the page geometry
\pagestyle{fancy} % options: empty , plain , fancy
\renewcommand{\headrulewidth}{0pt} % customise the layout...
\lhead{}\chead{}\rhead{}
\lfoot{}\cfoot{\thepage}\rfoot{}

%%% SECTION TITLE APPEARANCE
\usepackage{sectsty}
\allsectionsfont{\sffamily\mdseries\upshape} % (See the fntguide.pdf for font help)
% (This matches ConTeXt defaults)
%%% ToC (table of contents) APPEARANCE
\usepackage[nottoc,notlof,notlot]{tocbibind} % Put the bibliography in the ToC
\usepackage[titles,subfigure]{tocloft} % Alter the style of the Table of Contents
\renewcommand{\cftsecfont}{\rmfamily\mdseries\upshape}
\renewcommand{\cftsecpagefont}{\rmfamily\mdseries\upshape} % No bold!

%%% END Article customizations

%%% The "real" document content comes below...

\title{Software Engineering Project Weekly Report\\ \textbf{3D-KORN} \\ University of Bourgogne}
%\title{Seeded Image Segmentation using Laplacian Coordinates}
\author{Luca Canalini \and Ezequiel De La Roza \and Benjamin Lalande Chatin \and Roberto Paolella \and Umamaheswaran Raman Kumar \and Savinien Bonheur \and Albert Clerigues Garcia \and Daniel Gonzalez Dell \and Nayee Muddin Khan Dousai \and Pamir Ghimire \and Di Meng
}

\date{Oct 10, 2016} % Activate to display a given date or no date (if empty),
%         % otherwise the current date is printed 

\begin{document}
\maketitle
\newpage
\tableofcontents
\newpage

\section{Tasks completed}
\begin{itemize}
\item Infrastructure setup which enables us to work as a team
	\begin{itemize}
		\item Github for code repository and file sharing. \\Link for the repository -					https://github.com/umaatgithub/3D-KORN
		\item Trello for creating storyboards and setting up meetings
		\item Facebook group and messenger group for communication
		\item QT for development platform	
	\end{itemize}

\item Decided to use Kinect as the sensor for scanning and PCL(Point Cloud Library) for point cloud processing.
\item Basic UI designs and use cases are done and available in github, which are yet to be reviewed by all the members.
\item PCL installation
\end{itemize}

\section{Work in Progress}
We have divided the team into 3 groups and working on the POC(Proof of Concept) to understand and share the knowledge on the below set of topics. The papers read and the codes developed related to these are shared in the POC folder in Github. 

\begin{itemize}
\item Interfacing with Kinect sensor~\\
~\\Tasks:
\begin{itemize}
\item Understanding KinectFusion and DynamicFusion algorithms
\item  Control sensor operations(Eg: start scan, change angle, zoom in/out)
\item  Generate point cloud from scan
\end{itemize}
Members: Pamir, Dani, Umamaheswaran, Eze

\item Operations on point cloud~\\
~\\Tasks:
\begin{itemize}
\item Convert point cloud to mesh
\item Editing point cloud/mesh
\end{itemize}
Members: Clement, Luca, Roberto, Nayeem, Meng
	

\item User Interface~\\
~\\Tasks:
\begin{itemize}
\item Understand PCL library functions for handling mesh display in Qt
\item Rough UI design
\end{itemize}
Members: Albert, Benjamin
\end{itemize}	

\section{Tasks for this week}
\begin{itemize}
\item Complete the POC's and come up with a stable high level design(UseCase diagram) and low level design(Class diagram).
\end{itemize}

\end{document}
