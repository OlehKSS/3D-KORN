% !TEX TS-program = pdflatex
% !TEX encoding = UTF-8 Unicode

% This is a simple template for a LaTeX document using the "article" class.
% See "book", "report", "letter" for other types of document.

\documentclass[11pt]{article} % use larger type; default would be 10pt

\usepackage[utf8]{inputenc} % set input encoding (not needed with XeLaTeX)

%%% Examples of Article customizations
% These packages are optional, depending whether you want the features they provide.
% See the LaTeX Companion or other references for full information.
\usepackage{hyperref}
\usepackage{float}
%%% PAGE DIMENSIONS
\usepackage[margin=1.0in]{geometry} % to change the page dimensions
\geometry{a4paper} % or letterpaper (US) or a5paper or....
% \geometry{margin=2in} % for example, change the margins to 2 inches all round
% \geometry{landscape} % set up the page for landscape
%   read geometry.pdf for detailed page layout information

\usepackage{graphicx} % support the \includegraphics command and options

% \usepackage[parfill]{parskip} % Activate to begin paragraphs with an empty line rather than an indent

%%% PACKAGES
\usepackage{booktabs} % for much better looking tables
\usepackage{array} % for better arrays (eg matrices) in maths
\usepackage{paralist} % very flexible & customisable lists (eg. enumerate/itemize, etc.)
\usepackage{verbatim} % adds environment for commenting out blocks of text & for better verbatim
\usepackage{subfig} % make it possible to include more than one captioned figure/table in a single float
% These packages are all incorporated in the memoir class to one degree or another...
\setlength{\parskip}{0.5em}

%%% HEADERS & FOOTERS
\usepackage{fancyhdr} % This should be set AFTER setting up the page geometry
\pagestyle{fancy} % options: empty , plain , fancy
\renewcommand{\headrulewidth}{0pt} % customise the layout...
\lhead{}\chead{}\rhead{}
\lfoot{}\cfoot{\thepage}\rfoot{}

%%% SECTION TITLE APPEARANCE
\usepackage{sectsty}
\allsectionsfont{\sffamily\mdseries\upshape} % (See the fntguide.pdf for font help)
% (This matches ConTeXt defaults)
%%% ToC (table of contents) APPEARANCE
\usepackage[nottoc,notlof,notlot]{tocbibind} % Put the bibliography in the ToC
\usepackage[titles,subfigure]{tocloft} % Alter the style of the Table of Contents
\renewcommand{\cftsecfont}{\rmfamily\mdseries\upshape}
\renewcommand{\cftsecpagefont}{\rmfamily\mdseries\upshape} % No bold!

%%% END Article customizations

%%% The "real" document content comes below...

\title{Software Engineering Project Weekly Report\\ \textbf{3D-KORN} \\ University of Bourgogne}
%\title{Seeded Image Segmentation using Laplacian Coordinates}
\author{Luca Canalini \and Ezequiel De La Rosa \and Benjamin Lalande Chatin \and Roberto Paolella \and Umamaheswaran Raman Kumar \and Savinien Bonheur \and Albert Clerigues Garcia \and Daniel Gonzalez Dell \and Nayee Muddin Khan Dousai \and Pamir Ghimire \and Di Meng
}

\date{Oct 24, 2016} % Activate to display a given date or no date (if empty),
%         % otherwise the current date is printed 

\begin{document}
\maketitle
\newpage

\section{Tasks completed}
\begin{itemize}
\item Creation of an Excel file for task allocation and working progress (\url{https://goo.gl/WDHEjf)}. % A CORREGIR
\begin{itemize}
		\item List of activities being conducted. 
		\item Human resources allocated for each task.
		\item Deadlines establishment for task tracking.	\end{itemize}

\item Completed Gantt chart and created set of milestones.

%INSERT GANT CHART &&&&& MILESTONES! 
	
\item Knowledge transfer: KT session on coding standars was conducted.

\item Issue resolved:  Successfully interfaced with Kinect V2. Consequently, the build configuration for the project will switch to PCL 1.8 (source: (\url{http://unanancyowen.com/}) with MSCV 2015 and Qt 5.7.  
			
\item Acquisition of two working data sets of 3D pointclouds. Storage and sharing for POC tests in GitHub repository.
	\begin{itemize}
		\item Acquisition of pointclouds from the camera already PCL data-type.
	\end{itemize}   

\item Successful interfaced with Arduino trough QT serial port library.


\end{itemize}

\section{Work in Progress}

\begin{itemize}
\item Initial, preliminary versions of use-case and class diagrams shared in Github.
\item Advances in POC tests for 3D pointcloud cropping, segmentation and registration.
\item Scanner design. Horizontal, inter-team design agreement.
		\begin{itemize}
		\item BIM (bill of materials) waiting for decition.
	\end{itemize}  
\end{itemize}  	

\section{Tasks for this break}
For this special situation in which members of the group will be working remotely in the project,  we suggest the following goals 
aiming to prepare the coding step of the project:
\begin{itemize}
\item Finish the ongoing POC. 
\item Start implementing classes from the finished POC.
\item Finally, start building the basic structures of the projects code.

\end{itemize}


For achieving this, we suggest a team distribution an task allocation as follows:

\begin{itemize}
\item Project Initialization\\
~\\Tasks:
\begin{itemize}
\item Lay the basis for the projects repository and software architecture.
\item Start GUI POC with dummy implementation.
\end{itemize}
Members: Umamaheswaran, Benjamin, Nayeem.

\item Scanning ~\\
~\\Tasks:
\begin{itemize}
\item Wrapping scanner in 3DKorn\_ ScannerDevice class. 
\item 3D registration POC.
\item Research on Arduino, Qt and motors controlling for turntable.
\end{itemize}
Members: Albert, Ezequiel, Daniel, Pamir
	

\item Operations With Pointclouds~\\
~\\Tasks:
\begin{itemize}
\item Complete ongoing POC in segmentation and plane cropping.
\item Implement class 3DKorn\_ SegmentPointcloud, 
\end{itemize}
Members: Luca, Roberto, Savinien.


\item Training Group~\\
~\\Tasks:
\begin{itemize}
\item Continue practice in programming and Qt to consolidate knowledge. With that in mind we suggest reviewing and understanding with the help of other team members some of the Software Engineering labs. 
\end{itemize}
Members: Nayeem, Meng
\end{itemize}
\end{document}
