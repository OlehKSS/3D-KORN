% !TEX TS-program = pdflatex
% !TEX encoding = UTF-8 Unicode

% This is a simple template for a LaTeX document using the "article" class.
% See "book", "report", "letter" for other types of document.

\documentclass[11pt]{article} % use larger type; default would be 10pt

\usepackage[utf8]{inputenc} % set input encoding (not needed with XeLaTeX)

%%% Examples of Article customizations
% These packages are optional, depending whether you want the features they provide.
% See the LaTeX Companion or other references for full information.
\usepackage{hyperref}
\usepackage{float}
%%% PAGE DIMENSIONS
\usepackage[margin=1.0in]{geometry} % to change the page dimensions
\geometry{a4paper} % or letterpaper (US) or a5paper or....
% \geometry{margin=2in} % for example, change the margins to 2 inches all round
% \geometry{landscape} % set up the page for landscape
%   read geometry.pdf for detailed page layout information

\usepackage{graphicx} % support the \includegraphics command and options

% \usepackage[parfill]{parskip} % Activate to begin paragraphs with an empty line rather than an indent

%%% PACKAGES
\usepackage{booktabs} % for much better looking tables
\usepackage{array} % for better arrays (eg matrices) in maths
\usepackage{paralist} % very flexible & customisable lists (eg. enumerate/itemize, etc.)
\usepackage{verbatim} % adds environment for commenting out blocks of text & for better verbatim
\usepackage{subfig} % make it possible to include more than one captioned figure/table in a single float
% These packages are all incorporated in the memoir class to one degree or another...
\setlength{\parskip}{0.5em}

%%% HEADERS & FOOTERS
\usepackage{fancyhdr} % This should be set AFTER setting up the page geometry
\pagestyle{fancy} % options: empty , plain , fancy
\renewcommand{\headrulewidth}{0pt} % customise the layout...
\lhead{}\chead{}\rhead{}
\lfoot{}\cfoot{\thepage}\rfoot{}

%%% SECTION TITLE APPEARANCE
\usepackage{sectsty}
\allsectionsfont{\sffamily\mdseries\upshape} % (See the fntguide.pdf for font help)
% (This matches ConTeXt defaults)
%%% ToC (table of contents) APPEARANCE
\usepackage[nottoc,notlof,notlot]{tocbibind} % Put the bibliography in the ToC
\usepackage[titles,subfigure]{tocloft} % Alter the style of the Table of Contents
\renewcommand{\cftsecfont}{\rmfamily\mdseries\upshape}
\renewcommand{\cftsecpagefont}{\rmfamily\mdseries\upshape} % No bold!

%%% END Article customizations

%%% The "real" document content comes below...

\title{Software Engineering Project Weekly Report\\ \textbf{3D-KORN} \\ University of Bourgogne}
\author{Luca Canalini \and Ezequiel De La Rosa \and Benjamin Lalande Chatin \and Roberto Paolella \and Umamaheswaran Raman Kumar \and Savinien Bonheur \and Albert Clerigues Garcia \and Daniel Gonzalez Adell \and Nayee Muddin Khan Dousai \and Pamir Ghimire \and Di Meng
}

\date{Oct 31, 2016} % Activate to display a given date or no date (if empty),
%         % otherwise the current date is printed 

\begin{document}
\maketitle
\newpage

\section{Work in Progress}

\begin{itemize}

\item New tasks allocated for the vacation. For updates on individual tasks kindly refer to important links section.

\item Working on 3 development branches namely gui, sensor and pointcloud for independent development of tasks allocated.
\begin{itemize}
\item \textit{QT Graphical User Interface related classes - `gui' branch}\\
\textbf{Classes}: TDK\_MainWindow, TDK\_PointCloudScanWindow, TDK\_MenuBar, \\TDK\_StatusBar, TDK\_ToolBar, TDK\_DisplayWidget\\
\textbf{Members}: Umamaheswaran, Benjamin, Nayeem, Meng

\item \textit{Sensors and Scanning related classes - `sensor' branch}\\
\textbf{Classes}: TDK\_ArduinoController, TDK\_KinectV2Controller\\
\textbf{Members}: Albert, Ezequiel, Daniel, Pamir

\item \textit{Pointcloud operations related classes - `pointcloud' branch}\\
\textbf{Classes}: TDK\_PointCloudOperations, TDK\_MeshOperations, TDK\_PointCloudManager\\
\textbf{Members}: Luca, Roberto, Savinien, Umamaheswaran
\end{itemize}

\item Training Activities\\
Tune up specific programming skills for tasks in progress. Horizontal help and knowledge transfer between members of the different teams.\\ 
Members: Ezequiel, Nayeem, Meng, Roberto
\end{itemize}

\section{Important links}
\begin{itemize}
\item Task allocation and progress  (\url{https://goo.gl/WDHEjf)}
\item Github repository (\url{https://github.com/umaatgithub/3D-KORN})
\end{itemize}
\end{document}


